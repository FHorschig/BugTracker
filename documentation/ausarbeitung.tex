% Muster für die Seminarausarbeitung
% HPI Potsdam

\documentclass[11pt, a4paper]{article}

\usepackage{ngerman}
\usepackage[utf8]{inputenc} %Korrekte Kodierung der Umlaute nach UTF-8
\usepackage[T1]{fontenc} %Korrekte Kodierung der Umlaute nach UTF-8
\usepackage{amsfonts}
\usepackage{amssymb}
\usepackage{csquotes}
\usepackage{epsfig}   % Zum Einbinden von Bildern
\usepackage{url}      % Korrekter Satz von URLs
\usepackage{soulutf8}
\usepackage{color}    % Verwendung von Farben
\usepackage{listings} % Korrekter Satz von Listings und Quellcode


%Hilfs-Fonts - ohne Serifen (hier für Tabellen)
\newfont{\bib}{cmss8 scaled 1040}
\newfont{\bibf}{cmssbx8 scaled 1040}

\definecolor{lightgray}{gray}{0.85}

%Seitenformat-Definitionen
\topmargin0mm
\textwidth147mm
\textheight214mm
\evensidemargin5mm
\oddsidemargin5mm
\footskip19mm
\parindent=0in

\begin{document}          

\begin{titlepage}
  \begin{center} 
    \mbox{}
    \vspace{1cm}
    
    {\huge Annotating Digital Images of Insects \\[1em] {\LARGE Automatic Generation of RDF files using Computer Vision Methods}}  
        
    \vspace{4cm}
    
    Seminararbeit im Seminar \\[1em]
    {\large \sc Semantic Multimedia Technologies} \\[1em]
    Sommersemester 2015 \\[1em]
    Hasso-Plattner-Institut für Softwaresystemtechnik GmbH \\[1em]
    Universität Potsdam
    
    \vspace{4cm}
    
		vorgelegt von
		
    \vspace{1em}
    
		{\Large Leander Neiß} \\
		{\Large Friedrich Horschig}\\
		{\Large Clemens Frahnow}\\
		{\Large Sten Ächtner}
		
    \vspace{4em}
    
    31th~August 2015
  \end{center}
\end{titlepage}


\setcounter{page}{1}

% Zweite Seite = Kurzzusammenfassung
\begin{center}
{\bf Abstract} 
\end{center}

\noindent


\newpage

% Dritte Seite = Inhaltsverzeichnis
\tableofcontents 

\newpage

% Vierte Seite = Hier geht's eigentlich richtig los
\section{Introduction}
\label{sec_introduction}

The \emph{Museum für Naturkunde}, a natural history museum in Berlin, Germany, features an insect collection consisting of several million insect specimen.
This collection is digitalized in the context of the museum's EoS project.
Typically, the resulting images show multiple insect boxes  filled with insect specimen of the same species.
On average, such an image is about 100 MB in size and contains about 50 specimen.

In this paper, we present BugTracker, a tool developed to annotate the museum's insect collection images with information about the location of each single specimen in the image and what type of insect it is.
This should support the museum's work with those rather large images in order to use and further process them.

The paper presents an approach based on template matching that allows the automatic annotation of insect images with semantic information.
The algorithm itself is a multi-staged approach consisting of four main subtasks:

\begin{description}
    \item[Contour detection] to automatically generate template candidates and extract one of them as the actual template
    \item[Template matching] to identify the location of all insect specimen in the image and calculate their bounding box
    \item[QR code analysis] to identify qr codes, extract a species identifier, and lookup the biological information associated with it from a reference data set
    \item[Annotation] of the image by creating the output rdf file
\end{description}

This work is created in the context of the \emph{Semantic Multimedia Technologies} seminar during the summer semester 2015 at the Hasso Plattner Institute, University of Potsdam, Germany.

The remainder of this paper is structured as follows. 
Section \ref{sec_related} highlights related work.
The motivation for this work is presented in section \ref{sec_motivation} and detailed information about the implementation are described in the next section.
The results are evaluated in section \ref{sec_eval}.
In the last section we draw a conclusion and discuss it.
 
\newpage
%
\section{Aufbau und Inhalt der Seminararbeit}
\label{sec_aufbau}

Im vorangegangenen Kapitel hatten wir bereits die Gliederung einer Seminararbeit kurz vorgestellt und erläutert, welche inhaltlichen Punkte in der \glqq Einleitung\grqq\, behandelt werden sollten.
Die folgenden Abschnitte skizzieren inhaltlich die übrigen der bereits genannten Gliederungspunkte.

\subsection{Verwandte Arbeiten und wissenschaftlicher Hintergrund (Related Work)}
%%
Hier sind vor allem zwei inhaltliche Punkte zu berücksichtigen:
\begin{itemize}
\item {\bf Notwendige Vorarbeiten und Grundlagen, die zum Verständnis der Arbeit notwendig sind}\\
Keine bzw. kaum eine Arbeit beginnt als \glqq tabula rasa\grqq , d.h. meist bauen wir auf  vorhandenen Grundlagen bzw. Vorarbeiten auf.
Die zum Verständnis der eigenen Arbeit notwendigen Grundlagen und Voraussetzungen müssen in diesem Kapitel skizziert bzw. zusammengefasst werden.
Dabei sollte man vom durchschnittlichen Kenntnisstand eines Informatikers ausgehen, d.h. Allgemeinplätze und allzu Grundlegendes hat hier nichts zu suchen.
Genauso soll hier nicht notwendigerweise eine kompletter Wissenschaftszweig in epischer Tiefe ausgebreitet werden, sondern lediglich die zum Verständnis notwendigen Teilbereiche in skizzenhafter Form und mit Angabe von Literaturhinweisen zusammengefasst werden. 

\smallskip

Zum Beispiel können hier die Grundlagen und Vorzüge von Linked Open Data erläutert werden.

\smallskip

\item {\bf Alternative Ansätze und ggfs. Forschungsarbeiten zum Thema}\\
Besonders wichtig ist es, spezielle Vorarbeiten und alternative Ansätze zum behandelten Thema darzulegen.
Gibt es zu der von Ihnen gewählten Problemstellung alternative Lösungen, die einen anderen oder vergleichbaren Ansatz verfolgen? Wie unterscheiden sich diese Lösungen von Ihrem Vorschlag, wo liegen Vor- und Nachteile des jeweiligen Ansatzes? Grenzen Sie ihren eigenen Ansatz von den alternativen referenzierten Ansätzen argumentativ ab.

\smallskip

So könnten Sie erwähnen, dass Ihr eigener Ansatz z.B. von den selben Voraussetzungen ausgeht wie Ansatz XY, im Gegensatz zu diesem aber auf dem Einsatz von Linked Data Technologien beruht, und daher bessere / genauere / umfangreichere Ergebnisse erzielt. (Bitte wiederholen Sie diese Argumentation nicht. Sie dient nur als illustrierendes Beispiel.)

Wichtig ist, dass Sie jede der vorgestellten, alternativen Arbeiten 
\begin{itemize}
\item korrekt zitieren (Bibliografie),
\item kurz die wichtigsten Ergebnisse bzw. Strategien skizzieren und
\item diese (kurz und knapp) in Zusammenhang mit ihrer eigenen Arbeit stellen. 
\end{itemize}
Wie unterscheidet sich der eigene Ansatz von den vorgestellten Arbeiten? 
Warum ist der eigene Ansatz eventuell erfolgsversprechender? 

\end{itemize}

\subsection{Eigener Ansatz zur Lösung der gestellten Aufgabe (Method and Approach)}
%%
Hier haben Sie die Freiheit, Ihren eigenen Arbeiten angemessen viel Raum zur Verfügung zu stellen.
Achten Sie dabei auf einen logischen Aufbau der Darstellung, d.h. Grundlegendes zuerst.
\begin{itemize}
\item Wie sind Sie vorgegangen?
\item Wo gab es Probleme?
\item Wie wurden diese gelöst?
\item Schreiben Sie in verständlicher Weise und drücken Sie sich dabei jeweils möglichst präzise, d.h. unmissverständlich aus (vgl. Kap.~\ref{sec_stil})
\item Verwenden Sie Abbildungen, Tabellen und Beispiele, um Ihren Ansatz besser zu erläutern.
\item Setzen Sie kein Wissen als implizit vorhanden voraus, sondern sprechen Sie explizit alle Probleme und wichtigen Fakten an.
\item Wichtig: Was Sie hier nicht beschreiben, können wir nicht bewerten!
\end{itemize}
Bedenken Sie dabei stets, dass ein Leser nicht dasselbe Wissen besitzen kann wie Sie und dass Sie ihm deshalb ihre Ergebnisse erklären müssen.

Verwenden Sie bei der Darstellung Ihres Lösungsansatzes eine möglichst einfache Sprache.
Vermeiden Sie zahlreiche Schachtelungen und Nebensätze. Der dargestellte Sachverhalt ist meist bereits hinreichend komplex. 
Die zu seiner Darstellung verwendete Sprache sollte sollte den Zugang für den Leser nicht auch noch erschweren.


\subsection{Evaluation des wissenschaftlichen Beitrags (Evaluation)}
%
Viele wissenschaftliche Aufgabenstellungen erfordern den Nachweis der Qualität bzw. der Effizienz des vorgestellten Lösungsansatzes, d.h. das entwickelte Verfahren bzw. die vorgestellte technische Lösung muss objektiv getestet und anschließend bewertet werden.
Man unterscheidet hier grundsätzlich zwischen quantitativer und qualitativer Evaluation.
Während in der qualitativen Evaluation oft menschliche Testpersonen die Qualität einer Lösung entsprechend vorgegebener Qualitätskriterien beurteilen müssen, stützt sich die quantitative Evaluation auf vorgegebene Testdatensätze und einen Soll-/Ist-Vergleich.
Testdatensätze (auch Benchmarks) enthalten korrekte, oft manuell verifizierte Ergebnisse zu vorgegebenen Aufgabenstellungen.
Diese werden quantitativ mit den Ergebnissen des von Ihnen implementierten Verfahrens verglichen und in Form von statistischen Kenngrößen (z.B. Recall und Precision) beschrieben.

\begin{itemize}
\item Verwenden Sie (wenn möglich) vorgegebene Standard Benchmarks, um eine möglichst breite Vergleichbarkeit und Nachvollziehbarkeit zu gewährleisten.
\item Wenn Sie eigene Testdaten bzw. Benchmarks zusammenstellen, stellen Sie diese öffentlich (im WWW) zur Verfügung, um bessere Vergleichberkeit und Nachvollziehbarbeit zu gewährleisten.
\end{itemize}

\bigskip

Nicht jede Aufgabenstellung ist für eine solche Evaluation geeignet. Theoretische und mathematische Aufgabenstellungen resultieren oft in Lösungen, deren Gültigkeit mit mathematischen Methoden (mathematischen Beweisverfahren) belegt wird.



\subsection{Diskussion der erzielten Ergebnisse (Conclusion and Outlook)}
%%
In diesem Kapitel sollten Sie Ihre erzielten Ergebnisse präsentieren und deren Qualität diskutieren.
Dabei sollten (falls jeweils zutreffend) folgende Fragen beantwortet werden:
\begin{itemize}
\item Was wurde erreicht, was kann noch verbessert werden bzw. wo gibt es noch offene (evtl. aus Zeitgründen nicht implementierte) Punkte?
\item Sind die erzielten Ergebnisse objektiv? Gibt es Gründe, daran zu zweifeln?
\item Warum ist der eigene Ansatz besser/schlechter als die zum Vergleich herangezogenen?
\item Welche Vorbedingungen könnten verändert werden, um eventuell bessere Ergebnisse zu erzielen?
\item Wenn die Evaluation nicht aussagekräftig genug ist, wie könnte man sie noch verbessern?
\item Was haben Sie aus dem Seminar mitgenommen (z.B. Wo liegen die Vorteile der im Seminar behandelten wissenschaftlichen Methoden und Verfahren?)
\end{itemize}


\subsection{Zusammenfassung und Ausblick}

In diesem Abschnitt sollten die erzielten Ergebnisse noch einmal kurz zusammengefasst werden und ein Ausblick auf weiterführende Entwicklungsarbeiten gegeben werden (vgl. Kap.~\ref{sec_conclusion}). Hier können Sie z.B. ausführen, welche Arbeiten Sie aus Zeitgründen nicht mehr umsetzen konnten, aber für wichtig oder sinnvoll erachten. 
Unterscheiden Sie, welche Verbesserungsmaßnahmen kurzfristig bzw. langfristig vorgenommen werden könnten und versuchen Sie den dadurch erzielten Gewinn bzw. Vorteil zu quantifizieren.
\newpage
%
\section{Requirements}
\label{sec_requirements}



\newpage
%!TEX root = ../ausarbeitung.tex
\section{Concepts}
\label{sec_concepts}

\subsection{Edge Detection}
Edge Detection is a part of segmentation in image processing.
It is used to isolate some areas of an image from others, such as shapes in the foreground from the background.
An edge is determined by the difference of its adjacent brightness value.
The points where the image brightness has discontinuities are represented as curved line segments, the \textit{edges}.
A threshold image of the original image is computed to determine whether two brightness values are classified as a discontinuity.
Using a global, fixed threshold would classify everything as an edge, that has one adjacent brightness above, and one adjacent brightness below a specific, predefined value.
As second possibility there is an adaptive threshold.
It sets the threshold adaptively for each area in the image, using the mean brightness of that area.


\subsection{Template Matching}
Template Matching in image processing basically is a method to extract parts of an image that match with a chosen template based on a comparison function.
The algorithm takes all possible template sized sub-images and uses the comparison function to yield a value for the similarity to the template.
Based on this value, the best match can be found, or giving a threshold value x, all results with a higher or lower value than x can be gained.
There are multiple comparison functions like pixel-wise comparison or a simple perfect-match function.

The most relevant functions used in this paper are Correlation Coefficient and Histogram of Oriented Gradients which are introduced in the following sections.

\subsubsection{Correlation Coefficient}
The Correlation Coefficient Function (short ``CCOEFF'') first calculates the mean of all pixel values of the template as well as the mean of the pixel values in the selected sub-image.
By iterating over all pixels in the template, two sub-values are calculated per iteration:
The first one is the signed distance of the pixel value in the template to the mean value of the template.
The other one is the signed distance of the value in the sub-image at the corresponding pixel position to the mean value of the sub-image.
Both sub-values are multiplied in each iteration and the resulting values of all iterations are summed up.
To have a normed value for the comparison, the result is divided by the highest possible value.
Since the used parameters for the function are the distances to the mean, the function calculates a value independent of the absolute values.

The resulting formula for this function can be seen here:
\[ccoeff = \frac{\sum_{x,y} (T(x,y) \times I(x',y'))}{\sqrt{\sum_{x,y} (T(x,y)^2 \times \sum_{x,y} I(x',y')^2)}} \]
Where T(x,y) is the pixel value difference from the mean in the template at position x, y and I(x', y') the corresponding pixel value difference from the mean in the sub-image.

\subsubsection{Histogram of Oriented Gradients}
Going away from the pure pixel values, the Histogram of Oriented Gradients function (short ``HOG'') is based on gradient directions.
It counts the number of gradient orientations in each (self defined) section of the sub-image and compares the result with that from the template.
\cite{hog_function}.

\subsection{Machine Learning}

\subsection{Comparing Effectiveness of Algorithms}
To have an objective measurement of the effectiveness of changes in the algorithm, we will consider two established factors for data quality. 
They are based on a gold standard. 
The gold standard will be a manually selected and annotated subset of the given data. 
All algorithms will be tested against the same set of data and compared against the standard.
Some data will be rightfully found (true positives), some will be wrongfully found (false positives) and some insects won't be found even if they should have been (false negatives)

The first factor is the recall that measures how many relevant results were retrieved. 
This is achieved by dividing the true positives by all relevant results (true positives and true negatives).

The second factor is precision of the results how many of the found results were relevant. 
This is achieved by dividing the the true positives by all positives (true and negative).

To raise the meaning of each factor, they are usually combined in a harmonic mean called F-Measure. 
The resulting value can only be maximized by maximizing both values. 
A very low value in one of the factors will result in an overall low factor. 
The F-Measure is calculated as follows: 
\[
F-Measure = 2*\frac{Precision*Recall}{Precision+Recall}
\]


\newpage
%!TEX root = ../ausarbeitung.tex
\section{Implementation}
\label{sec_implementation}

We implemented BugTracker as commandline-based Python tool.
It takes an image as input, and produces an RDF file with annotations of the found insects.
Template

\subsection{Contour Detection}
As a basic approach, we use Contour Detection to find insects in a given image.
The input image converted to gray-scale is used to detect edges as mentioned in Section \ref{sec_concepts}.
We use an adaptive threshold to create a binary image, which is then transformed by a morphological operation from OpenCV, in order to close gaps in areas of interest.
On this transformed binary image, an OpenCV contour detection function is applied, which groups connected edges into one contour.
Those contours are a first approach on finding something on an image.
However, the threshold and edge detection algorithm can not possibly know, what we are looking for.
The found contours can be anything that appears to stand out from the background, not only insects.
This method is a good approach to just find anything on the image, but not precise enough for our needs.


\subsection{Template Matching}

\subsection{Automated Template Extraction}


\subsection{Annotations}
The simplest annotation is just an RDF-triple locating a bug on an image and describing it as Organism. 
The definition for an Organism and all properties we use to describe are part of the Darwin Core \footnote{http://rs.tdwg.org/dwc/} Standard of describing living organisms.

We can extract further information from a CSV file that was provided by the Natural History Museum of Berlin.
It contains a general overview of all species in a photograph. 
This includes dates and information about the family.

Whenever a new bug is found, it is added to the collection of bugs that will be written as RDF file at the end.
It is possible to annotate every bug with additional, specific properties (apart from those that apply to every insect in the box).

Such information can be found in QR codes as described in the next section.

\subsubsection{QR Code Analysis}

\subsection{Integrating Benchmarking}
Due to the availability of each algorithm as single step in the pipeline, it was easy to execute them in the exact same environment with the exact same parameters. 
Different from the usual pipeline, the step of selecting the examples was automated and the step of writing out RDF files was replaced by analyzing the discovered bugs.

To have a reliable set of data (a ``gold standard''), we created annotations for some photos manually. 
The photos were mainly taken randomly. 
In addition, we chose some photos which we found hard to analyze (due to transparent wings, difficult shapes or different sizes).

The actual gold standard was not a set of RDF files but a collections of comma-separated values. 
These CSV files were easier to create (with a helping tool we wrote to manually annotate the bugs) and easier to analyze than RDF documents with same contents.
The actual results, their effect on our process and how we could optimize the benchmarking process will be discussed in chapter \ref{sec_conclusion}.
\newpage
%
\section{Evaluation and Discussion}
\label{sec_eval}
\label{sec_conclusion}

\subsection{Benchmarks}
To evaluate our tool, we created a benchmark set containing several images where we manually marked the insects.
The result of one benchmark run is expressed as \textit{Recall}, \textit{Precision}, and \textit{F-Measure} \cite{f_measure}.
Although F-Measure is a very common way to measure the quality of results, we consider changing  to another method that puts a higher weight on the quantity of results. 
Usually, it would be easier to refine existing results than to find additional insects. 
Also, when looking for false positives, it is possible that true positives are found because the bounding box did not fit correctly.
However, it is more useful to see an image of multiple insects or a cropped insect instead of no result at all.

The test set is crucial for the accuracy of the benchmark. 
It is always good to have a large test set to improve the accuracy of the benchmark.
Therefore continuous extending of the test set is needed. 

\subsection{Results}

Running the benchmark on our test set yields the following results:
%
\begin{description}
	\item[Recall] 0.82
	\item[Precision] 0.80
	\item[F-Measure] 0.81
\end{description}
%
The used method consists of the automated template extraction by contour detection, followed by a template matching with the extracted template.

In our test set we have images of well-shaped insects, like butterflies, and different bugs,
but also images of border cases, like insects with transparent wings, stick insects, or insects which are missing some parts of their body.
Those border cases cause a worsening of our benchmark results.
Using a slightly smaller test set, where we only use the common cases, we can achieve much better results:
%
\begin{description}
	\item[Recall] 0.90
	\item[Precision] 0.97
	\item[F-Measure] 0.94
\end{description}
%
\subsection{Discussion}
When executing the benchmark, there is an option to enable a mode to see the evaluated image.
By showing all false and true positives and true negatives it is possible to see what causes the resulting numbers.
The main problem for not having 100\% lies in the diversity of the insects:
Butterflies of the same class often have differing patterns on their wings.
Also differences in color and size are serious, especially between male and female ones of the same species.
Another problem is the condition of the insects:
Some animals have missing parts like wings or legs or they differ in the folding of their wings.
Furthermore sometimes the insects are aligned horizontally and sometimes vertically in one box, so that the template won't find all insects of one alignment.
The first box in Figure \ref{fig:diverse} shows such a high diversity like missing wings.

The condition of the box where the insects are in is another influence on the result.
Sometimes the background of the images has dirt on it, that makes it hard to find a difference to the insects, especially if they have transparent parts like in the second box of Figure \ref{fig:diverse}

\begin{center}
	\includegraphics[width=1\textwidth]{images/diversity_dirt.jpg}
	\captionof{figure}{High diversity in one box and dirt in the other one}
	\label{fig:diverse}
\end{center}
All these points make it hard to have a result close to 100\%.
But an F-measure of 94\% is still a very good result and even 81\% with a lot of edge cases is pretty good.

\subsection{Machine Learning}
As described in Section \ref{sec_concepts}, using an SVM is a Machine Learning approach on classification of images.
In the process of developing our tool, we also implemented a SVM, to see if it fits our purposes.
We did not use it in our final method, since a really big training data set is required, which we did not have at that time.
But given proper training data, an SVM would be a valid classification method.

\subsection{Future Work}
Since we can get the classification of the specimen in a picture out of the QR-code, it should be possible to improve the template extraction.
If there is an online-service for getting images of insects by their name, the result of the automated template extraction could be verified on a much more precise level.

Another idea is to have a feedback for the user to show him critical findings or pictures with no matches at all.
Furthermore before processing all images, the software could just generate all templates and show suspect ones to the user to verify if it is a good template or just a label or QR-Code.

In a separate preprocessing step it can be useful to remove all labels and QR-codes from the image before looking for a template.
We spend a lot of time at this, but found no satisfying results without deleting true positives.

As mentioned before, with the help of a large trainings set, Machine Learning could be an interesting method to verify findings or to improve the segmentation.
Such training sets could be gathered from other sources, where an annotation already has been done.



%Hier kommt das Literaturverzeichnis
\newpage

\addcontentsline{toc}{section}{Bibliography} % Zeile für das Inhaltsverzeichnis

\bibliography{bibfile}
\bibliographystyle{alphadin}

\end{document}
