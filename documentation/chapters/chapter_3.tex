%
\section{Motivation}
\label{sec_motivation}

The main idea of the BugTracker project is to support the digitalization of an insect collection by creating an automated tool that allows to annotate thousands of images.
Existing tools and algorithms are explored in order to reuse them, and eventually combined for the best results. The four aspects in detail:

\paragraph{Support digitalization}
The EoS project of the \emph{Museum für Naturkunde} in Berlin is aimed at digitalizing the museum's insect collection.
The BugTracker tool allows to support this process by providing additional information about the data such as the position of a single insect in an image.

Further, the QR codes in each image provide information about the taxonomic classification of those insects.
So far, the included species of each image are known but annotating each insect and linking it to QR code information offers more details about it.

\paragraph{Automatic Annotation Process}
The EoS project resulted in a few thousand images of insects.
In order to support annotating this amount of images, it is important to develop a tool that completely automates the annotation process.
This includes everything from reading an image, processing the image and its meta data, up to creating the rdf annotations.

\paragraph{Reuse Existing Tools and Algorithms}
The development of the BugTracker tool was part of a university project over the span of one semester. 
In turn, to create a well-suited tool in this amount of time, it is important to explore and reuse already existing tools and algorithms instead of implementing everything from scratch.

\paragraph{Combined Algorithm}
Single algorithms lead to great results for some images.
Unfortunately, the same algorithms fail for others kinds of images.
As single algorithms were not able to lead to satisfying results, different image processing algorithms were used for automatic template extraction and the following template matching.
This combined approach is supposed to leverage the advantages of more than one algorithm to yield better results.