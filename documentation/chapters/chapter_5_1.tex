%
\section{Evaluation}
\label{sec_eval}

\subsection{Comparing Effectiveness of Algorithms}
To have an objective measurement of the effectiveness of changes in the algorithm, we will consider two established factors for data quality. 
They are based on a gold standard. 
The gold standard will be a manually selected and annotated subset of the given data. 
All algorithms will be tested against the same set of data and compared against the standard.
Some data will be rightfully found (true positives), some will be wrongfully found (false positives) and some insects won't be found even if they should have been (false negatives)

The first factor is the recall that measures how many relevant results were retrieved. 
This is achieved by dividing the true positives by all relevant results (true positives and true negatives).

The second factor is precision of the results how many of the found results were relevant. 
This is achieved by dividing the the true positives by all positives (true and negative).

To raise the meaning of each factor, they are usually combined in a harmonic mean called F-Measure. 
The resulting value can only be maximized by maximizing both values. 
A very low value in one of the factors will result in an overall low factor. 
The F-Measure is calculated as follows: 
\[
F-Measure = 2*\frac{Precision*Recall}{Precision+Recall}
\]
