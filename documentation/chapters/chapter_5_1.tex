%
\section{Evaluation}
\label{sec_eval}

\subsection{Benchmarks}
Although F-Measure is a very common way to measure the quality of results, we consider changing  to another method that puts a higher weight on the quantity of results.
Usually, it would be easier to refine existing results than to find additional insects. 
Also, when eliminating bad insects, it is possible that rightfully found insects were removed just because the bounding box didn't fit correctly.
For the user of our results, it would be much more useful to see an image of multiple insects or a cropped insect instead of no result.

The test set is crucial for the accuracy of the benchmark. 
With few doubt, it would be better to extend the test set. 
The problem is that edge cases (like very small or very thin insects) might lose their importance when a majority of bugs has a similar shape.
We can't verify this grade of diversity as we had access to a subset of all images.

In case this assumption holds true, the benchmark should verify two test sets where one contains all edge cases.
