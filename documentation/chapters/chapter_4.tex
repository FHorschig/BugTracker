%
\section{Concepts}
\label{sec_concepts}

\subsection{Edge Detection}
Edge Detection is a part of segmentation in image processing.
It is used to isolate some areas of an image from others, such as shapes in the foreground from the background.
An edge is determined by the difference of its adjacent brightness value.
The points where the image brightness has discontinuities are represented as curved line segments, the \textit{edges}.
A threshold image of the original image is computed to determine whether two brightness values are classified as a discontinuity.
Using a global, fixed threshold would classify everything as an edge, that has one adjacent brightness above, and one adjacent brightness below a specific, predefined value.
As second possibility there is an adaptive threshold.
It sets the threshold adaptively for each area in the image, using the mean brightness of that area.


\subsection{Template Matching}
\subsubsection{Correlation Coefficient}
\subsubsection{Histogram of Oriented Gradients}
\subsection{Machine Learning}

