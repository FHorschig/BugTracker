%!TEX root = ../ausarbeitung.tex
\section{Concepts}
\label{sec_concepts}

TODO: Explain why machine learning is a hard-to-implement approach. Huge set of human-controlled training data and stuff.

\subsection{Contour Detection}
\subsection{Template Matching}
Template Matching in image processing basically is a method to extract parts of an image that match with a chosen template based on a comparison function.
The algorithm takes all possible template sized subimages and uses the comparison function to yield a value for the similarity to the template.
Based on this value, the best match can be found, or giving a threshold value x, all results with a higher or lower value than x can be gained.
There are multiple comparison functions like pixelwise comparison or a simple perfect-match function.
The most relevant functions used in this paper are Correlation Coefficient and Histogram of Oriented Gradients which are introduced in the following sections.

\subsubsection{Correlation Coefficient}
The Correlation Coefficient Function (short CCOEFF) first calculates the mean of all pixel values of the template as well as the mean of the pixel values in the selected subimage.
By iterating over all pixels in the template, two subvalues are calculated per iteration:
The first one is the signed distance of the pixelvalue in the template to the mean value of the template.
The other one is the signed distance of the value in the subimage at the correspondung pixelposition to the mean value of the subimage.
Both values are multiplied in each iteration and the resulting values of all iteraitons are sumed up.
To have a normed value for the comparison, the result is divided by the highest possible value.
Since the used parameters for the function are the distances to the mean, the function calculates a value independent of the absolut values. 
\subsubsection{Histogram of Oriented Gradients}

\subsection{Machine Learning}

