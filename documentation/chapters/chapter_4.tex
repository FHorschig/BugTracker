%!TEX root = ../ausarbeitung.tex
\section{Concepts}
\label{sec_concepts}

\subsection{Edge Detection}
Edge Detection is a part of segmentation in image processing.
It is used to isolate some areas of an image from others, such as shapes in the foreground from the background.
An edge is determined by the difference of its adjacent brightness value.
The points where the image brightness has discontinuities are represented as curved line segments, the \textit{edges}.
A threshold image of the original image is computed to determine whether two brightness values are classified as a discontinuity.
Using a global, fixed threshold would classify everything as an edge, that has one adjacent brightness above, and one adjacent brightness below a specific, predefined value.
As second possibility there is an adaptive threshold.
It sets the threshold adaptively for each area in the image, using the mean brightness of that area.


\subsection{Template Matching}
Template Matching in image processing basically is a method to extract parts of an image that match with a chosen template based on a comparison function.
The algorithm takes all possible template sized sub-images and uses the comparison function to yield a value for the similarity to the template.
Based on this value, the best match can be found, or giving a threshold value x, all results with a higher or lower value than x can be gained.
There are multiple comparison functions like pixel-wise comparison or a simple perfect-match function.

The most relevant functions used in this paper are Correlation Coefficient and Histogram of Oriented Gradients which are introduced in the following sections.

\subsubsection{Correlation Coefficient}
The Correlation Coefficient Function (short ``CCOEFF'') first calculates the mean of all pixel values of the template as well as the mean of the pixel values in the selected sub-image.
By iterating over all pixels in the template, two sub-values are calculated per iteration:
The first one is the signed distance of the pixel value in the template to the mean value of the template.
The other one is the signed distance of the value in the sub-image at the corresponding pixel position to the mean value of the sub-image.
Both sub-values are multiplied in each iteration and the resulting values of all iterations are summed up.
To have a normed value for the comparison, the result is divided by the highest possible value.
Since the used parameters for the function are the distances to the mean, the function calculates a value independent of the absolute values.

The resulting formula for this function can be seen here:
\[ccoeff = \frac{\sum_{x,y} (T(x,y) \times I(x',y'))}{\sqrt{\sum_{x,y} (T(x,y)^2 \times \sum_{x,y} I(x',y')^2)}} \]
Where T(x,y) is the pixel value difference from the mean in the template at position x, y and I(x', y') the corresponding pixel value difference from the mean in the sub-image.

\subsubsection{Histogram of Oriented Gradients}
Going away from the pure pixel values, the Histogram of Oriented Gradients function (short ``HOG'') is based on gradient directions.
It counts the number of gradient orientations in each (self defined) section of the sub-image and compares the result with that from the template.
\cite{hog_function}.

\subsection{Machine Learning}

