%!TEX root = ../ausarbeitung.tex
\section{Implementation}
\label{sec_implementation}

\subsection{Contour Detection}
\subsection{Template Matching}


\subsection{Annotations}
The simplest annotation is just an RDF-triple locating a bug on an image and describing it as Organism. 
The definition for an Organism and all properties we use to describe are part of the Darwin Core \footnote{http://rs.tdwg.org/dwc/} Standard of describing living organisms.

We can extract further information from a CSV file that was provided by the Deutsches Naturkundemuseum.
It contains a general overview of all species in a photograph. 
This includes dates and information about the family.

Whenever a new bug is found, it is added to the collection of bugs that will be written as RDF file at the end.
It is possible to annotate every bug with additional, specific properties (apart from those that apply to every insect in the box).

Such information can be found in QR codes as described in the next section.

\subsubsection{QR Code Analysis}

\subsection{Integrating Benchmarking}
Due to the availability of each algorithm as single step in the pipeline, it was easy to execute them in the exact same environment with the exact same parameters. 
Different from the usual pipeline, the step of selecting the examples was automated and the step of writing out RDF files was replaced by analyzing the discovered bugs.

To have a reliable set of data (a ``gold standard''), we created annotations for some photos manually. 
The photos were mainly taken randomly. 
In addition, we chose some photos which we found hard to analyze (due to transparent wings, difficult shapes or different sizes).

The actual gold standard was not a set of RDF files but a collections of comma-separated values. 
These CSV files were easier to create (with a helping tool we wrote to manually annotate the bugs) and easier to analyze than RDF documents with same contents.
The actual results, their effect on our process and how we could optimize the benchmarking process will be discussed in chapter \ref{sec_conclusion}.