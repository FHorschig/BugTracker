% Vierte Seite = Hier geht's eigentlich richtig los
\section{Introduction}
\label{sec_introduction}

The \emph{Museum für Naturkunde}, a natural history museum in Berlin, Germany, features an insect collection consisting of several million insect specimen.
This collection is digitalized in the context of the museum's EoS project.
Typically, the resulting images show multiple insect boxes  filled with insect specimen of the same species.
On average, such an image is about 100 MB in size and contains about 50 specimen.

In this paper, we present BugTracker, a tool developed to annotate the museum's insect collection images with information about the location of each single specimen in the image and what type of insect it is.
This should support the museum's work with those rather large images in order to use and further process them.

The paper presents an approach based on template matching that allows the automatic annotation of insect images with semantic information.
The algorithm itself is a multi-staged approach consisting of four main subtasks:

\begin{description}
    \item[Contour detection] to automatically generate template candidates and extract one of them as the actual template
    \item[Template matching] to identify the location of all insect specimen in the image and calculate their bounding box
    \item[QR code analysis] to identify qr codes, extract a species identifier, and lookup the biological information associated with it from a reference data set
    \item[Annotation] of the image by creating the output rdf file
\end{description}

This work is created in the context of the \emph{Semantic Multimedia Technologies} seminar during the summer semester 2015 at the Hasso Plattner Institute, University of Potsdam, Germany.

The remainder of this paper is structured as follows. 
Section \ref{sec_related} highlights related work.
The motivation for this work is presented in section \ref{sec_motivation} and detailed information about the implementation are described in the next section.
The results are evaluated in section \ref{sec_eval}.
In the last section we draw a conclusion and discuss it.
